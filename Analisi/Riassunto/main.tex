%https://github.com/lucach/appunti-unimib/blob/master/Analisi/main.tex

\documentclass[a4paper,12pt]{book}

\usepackage{fullpage}
\usepackage[italian]{babel}
\usepackage[utf8]{inputenc}
\usepackage{amssymb}
\usepackage{amsthm}
\usepackage{graphics}
\usepackage{amsmath}
\usepackage{amstext}
\usepackage{engrec}
\usepackage{rotating}
\usepackage[safe,extra]{tipa}
\usepackage{showkeys}
\usepackage{multirow}
\usepackage{hyperref}
\usepackage{microtype}
\usepackage{enumerate}
\usepackage{braket}
\usepackage{marginnote}
\usepackage{pgfplots}
\usepackage{cancel}
\usepackage{polynom}


\usetikzlibrary{decorations.markings, intersections}
\usetikzlibrary{snakes}
\usetikzlibrary{arrows}


\frenchspacing
\pagestyle{plain}

% This is to suppress the 'Chapter N' title at the beginning of every
% chapter while keeping chapter numbers.
\usepackage[pagestyles]{titlesec}
\titleformat{\chapter}[display]{\normalfont\bfseries}{}{0pt}{\Huge}

\title{Riassunto Analisi Matematica}
\author{Cristian Baldi}


\newtheorem{tip}[theorem]{Spiegazione}

\begin{document}

\maketitle

\tableofcontents

\chapter{Prima di iniziare}

Un paio di cose prima di iniziare.

Questo riassunto è in preparazione per l'esame orale di Analisi Matematica Unimib Corso Informatica.

Fonte dei contenuti:
\begin{itemize}
\item Analisi Matematica, Paolo Maurizio Soardi
\item Appunti di Matematica, Luca Chiodini
\item WikiToLearn, Analisi 1
\end{itemize}


\input 2_Successioni.latex
\input 3_Serie_Numeriche.latex
\input 4_Funzioni.latex
\input 5_Derivate.latex
\input 6_Primitive.latex

\end{document}
