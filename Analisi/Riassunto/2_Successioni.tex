\chapter{Successioni}

\section{Successione numerica}

\subsection{Definizione}

Una successione è una funzione $x : N \rightarrow R$ indicabile anche con $\{X_{n}\}_{n \in N}$.

\begin{tip}
La successione associa ad ogni numero naturale $n$ un numero reale $a_{n}$.

Una successione numerica è una lista ordinata e infinita di numeri reali.
\end{tip}

\subsection{Successione limitata}

Una successione è detta superiormente limitata se esiste un $M$ tale che $a_{n} < M$ $\forall n \in N$

\begin{tip}
Una successione è detta limitata se esiste un numero $M$ che sovrasta tutti i termini della successione. 
\end{tip}

Una successione è detta inferiormente limitata se esiste un $m$ tale che $a_{n} > m$ $\forall n \in N$

\subsection{Successione Crescente}

Una successione $x_{n}$ è crescente se $x_{n+1} > x_{n}$ per ogni n.

\begin{tip}
Una successione è crescente se, preso un termine, il suo termine successivo è sempre più grande del termine corrente.

$\{ 0, 1, 2, 3, 4, 5,\ldots \}$
\end{tip}

\subsection{Successione Decrescente}

Una successione $x_{n}$ è decrescente se $x_{n+1} < x_{n}$ per ogni n.

\begin{tip}
Una successione è decrescente se, preso un termine, il suo termine successivo è sempre più piccolo del termine corrente. 

$\{ 0, -1, -2, -3, -4, -5, \ldots \}$
\end{tip}

\subsection{Successione Non Decrescente}

Una successione $x_{n}$ è non decrescente se $x_{n+1} >= x_{n}$ per ogni n.

\begin{tip}
Una successione è non decrescente se, preso un termine, il suo termine successivo è sempre o uguale o più grande del termine corrente.

$\{ 0, 1, 2, 2, 3, 5, 5, 5, \ldots \}$
\end{tip}

\subsection{Successione Non Crescente}

Una successione $x_{n}$ è non crescente se $x_{n+1} <= x_{n}$ per ogni n.

\begin{tip}
Una successione è non crescente se, preso un termine, il suo termine successivo è sempre o uguale o più piccolo del termine corrente.

$\{ 0, -1, -1, -3, -4, -4, \ldots \}$
\end{tip}

\subsection{Successione monotona}

Una successione è monotona se è crescente o decrescente o non crescente o non decrescente.

\section{Limite di una successione}

\subsection{Definizione}

$L \in R$ è il limite di $\{x_{n}\}$ se per ogni intorno  $\epsilon > 0$ esiste $N \in \mathbb{N}$ tale che, per ogni $n > N$, $$L - \epsilon < \{x_{n}\} < L + \epsilon$$

Tale limite si scrive anche come $$\lim_{n\to\infty} x_{n} = L$$ 

\begin{tip}
Un numero reale L è limite di una successione $\{x_{n}\}$ se la distanza fra i numeri $x_{n}$ e L è aribrariamente piccola quando n è sufficientemente grande.

Il limite di una successione è il valore a cui tendono i termini di una successione.
\end{tip}

\subsection{Successione convergente e divergente}

Se il limite esiste la successione è detta convergente, se il limite non esiste la successione è detta divergente.

\subsection{Successioni infinitesime }

Se una successione è convergente e il suo limite $L$ è 0, questa è detta infinitesima.

\section{Teorema di unicità del limite}

\textbf{Enunciato}

Sia $\{x_{n}\}$ una successione. Se $\{x_{n}\}$ ha limite $L$ e $\{x_{n}\}$ ha limite $L'$ allora $L=L'$

\begin{tip}
Se $\{x_{n}\}$ ha limite, questo è unico.

La successione $\{x_{n}\}$ non ammette due limiti diversi.
\end{tip}

\textbf{Dimostrazione}

Supponiamo che per assurdo $L \neq L'$ (cioè che la successione abbia due limiti diversi).

Prendiamo $\frac{\epsilon}{2}$ tale che $\epsilon < |L-L'|$

Per la definizione di limite esiste un $N$ tale che $|x_{n}-L| < \frac{\epsilon}{2}$ per ogni $n > N$, che si traduce in $x_{n} < L + \frac{\epsilon}{2}$ e $x_{n} > L - \frac{\epsilon}{2}$

Per l'assurdo imposto (l'esistenza dei due limiti) esiste nache un $N'$ tale che se $n > N'$ allora $|x_{n}-L'| < \frac{\epsilon}{2}$

Queste due condizioni si verificano entrambe nel seguente caso $|x_{n}-L| + |x_{n}-L'| < 2\frac{\epsilon}{2}$

$$|L-L'| = |x_{n}-L'| - |x_{n}-L| \leq |x_{n}-L| + |x_{n}-L'| < \epsilon$$

Quindi $\epsilon < |L-L'| < \epsilon$ che è ovviamente un assurdo, così da non verificare la nostra ipotesi iniziale.


\section{Teorema della permanenza del segno}


\textbf{Enunciato}

Una successione $\{x_{n}\}$ che converge ad un limite $L > 0$ (e quindi anche a $+\infty$) ha definitivamente soltanto termini positivi.

\begin{tip}
In altre parole, esiste un $N$ tale che $x_{n}>0$ per ogni $n > N$. 

Dopo un certo $N$ tutti i termini della successione sono positivi.
\end{tip}

\textbf{Dimostrazione}

Se il limite di $x_{n}$ è $L$ ed è finito, prendiamo $\epsilon = L$ e usiamolo nella definizione di limite.

Esiste quindi un $N$ tale che per ogni $n > N$ si ha che $|x_{n}-L| < \epsilon$ cioè
$$|x_{n}-L| < L$$
$$L - L < x_{n} < L + L$$
in particolare a noi interessa che 
$$L - L < x_{n}$$
cioè che $x_{n}>0$ per ogni $n > N$-

Se il limite di $x_{n}$ è infinito per la definizione di convergenza, dato un $M > 0$ qualsiasi, esiste $N$ tale che $x_{n}>M$ per ogni $n>N$.

\section{Teorema di esistenza del limite per successioni monotone}

\textbf{Enunciato}

Una successione monotona di numeri reali converge sempre ad un limite L.

Più precisamente, il limite di una successione crescente è il suo estremo superiore, mentre il limite di una successione decrescente è il suo estremo inferiore.

Il limite è finito solo se la successioni è limitata.

\textbf{Dimostazione}

Supponiamo $\{x_{n}\}$ monotona crescente.

Se la successione è illimitata, allora per ogni $M$ esiste un $N \in \mathbb{N}$ tale che $x_{N} > M$ (cioè esisterà un termine che prima o poi sarà più grande di M); di conseguenza, per la monotonia, $x_{n} > M$ per ogni $n > N$, quindi il limite di $\{x_{n}\}$ è infinito.

Se la successione è limitata, allora ha un estremo superiore $S$. Per la definizione di estremo superiore, per ogni $\epsilon > 0$ esiste un $N \in \mathbb{N}$ tale che $x_{N} > S - \epsilon$. Di conseguenza $x_{n} > S$ per ogni $n > N$, quindi S è il limite di $\{x_{n}\}$.

\section{Teorema del confronto per le successioni}

\textbf{Enunciato}

Prese $\{a_{n}\}$, $\{b_{n}\}$, $\{c_{n}\}$ tre successioni, tali che, definitivamente $a_{n} \leq b_{n} \leq c_{n}$ e 

$$lim_{n \to +\infty} a_{n} = lim_{n \to +\infty} c_{n} = L$$ 

allora 

$$lim_{n \to +\infty} b_{n} = L$$

\begin{tip}
È informalmente chiamato teorema dei due carabinieri, per un'allegoria: il teorema sarebbe rappresentato da due carabinieri (due funzioni o successioni a,c che si stringono sempre di più) che conducono in arresto un prigioniero (una funzione o successione b): questo tende sicuramente allo stesso punto dove tendono i carabinieri (il limite comune di a e c).
\end{tip}

\textbf{Dimostrazione}

Partiamo dalla definizione di limite, per ogni $\epsilon > 0$ esiste un $N$ (ed in questo caso anche un $N'$) tali che:

$$L-\epsilon < a_{n} < l+\epsilon \forall n>N$$

$$L-\epsilon < b_{n} < l+\epsilon \forall n>N'$$

Quindi per ogni $n$ maggiore di $M=max\{N,N'\}$ si ottiene la seguente:

$$L-\epsilon < a_{n} \leq b_{n} \leq c_{n} < L+\epsilon$$

Quindi per ogni $\epsilon>0$ eiste un $M$ tale che $L-\epsilon < b_{n} < L+\epsilon \forall n>M$

Cioè la successione $b_{n}$ ha limite $L$.

\section{Criterio del rapporto per successioni}

\textbf{Enunciato}

Sia $\{x_n\}$ una successione a termini positivi e sia 
\begin{equation*}
L = \lim_{n \to +\infty} \frac{x_{n+1}}{x_n}
\end{equation*}
Allora:
\begin{itemize}
\item se $L > 1$ la successione è definitivamente crescente e $\lim x_n = +\infty$.
\item se $0 \le L < 1$ la successione è definitivamente decrescente e $\lim x_n = 0$.
\end{itemize}


\textbf{Dimostrazione}
\begin{itemize}
\item se $L > 1$ allora possiamo imporre $L = 1 + 2\epsilon$. Per definizione di limite $\exists N$ tale che 
\begin{equation*}
\frac{x_{n+1}}{x_n} > L - \epsilon \qquad \forall n > N
\end{equation*}
\begin{equation*}
\frac{x_{n+1}}{x_n} > 1 + \epsilon \qquad \forall n > N
\end{equation*}

Quindi $x_{n+1} > x_n \cdot (1+\epsilon) > x_n$ per $n > N$. Quindi la successione è definitivamente crescente.

Proseguendo otteniamo:
\begin{align*}
x_{N+2} &> x_{N+1} \cdot (1+\epsilon) \\
x_{N+3} &> x_{N+2} \cdot (1+\epsilon) > x_{N+1} \cdot (1+\epsilon)^2 \quad \text { e così via\dots}
\end{align*}
Generalizzando:
\begin{equation*}
x_n > (1+\epsilon)^{n-(N+1)} \cdot x_{N+1}
\end{equation*}
Poiché $(1+\epsilon)^{n-(N+1)}$ diverge a $+\infty$, per il teorema del confronto anche $\lim x_n = +\infty$.

\item se $0 < L < 1$ procediamo in modo analogo al caso precedente. Imponiamo $L = 1 - 2\epsilon$. Per definizione di limite $\exists N$ tale che
\begin{equation*}
\frac{x_{n+1}}{x_n} < L + \epsilon \qquad \forall n > N
\end{equation*}
\begin{equation*}
\frac{x_{n+1}}{x_n} < 1 - \epsilon \qquad \forall n > N
\end{equation*}

Come prima vale:
\begin{equation*}
0 < x_n < (1-\epsilon)^{n-(N+1)} \cdot x_{N+1} \qquad \forall n>N
\end{equation*}

Per il criterio del confronto, essendo $\lim (1-\epsilon)^{n-(N+1)} \cdot x_{N+1} = 0$, allora $\lim x_n = 0$. Inoltre, $x_{n+1} < x_n \cdot (1 - \epsilon) < x_n$; quindi la successione è definitivamente decrescente.
\end{itemize}


\section{Algebra dei limiti }

Date due successioni $a_{n}$ e $b_{n}$ tali che 
\begin{itemize}
\item $\lim_{n \to inf} a_{n} = a$
\item $\lim_{n \to inf} b_{n} = b$
\end{itemize}

\begin{center}
$\lim_{n \to inf} a_{n} + b_{n} = a + b$ \qquad
$\lim_{n \to inf} a_{n} * b_{n} = a * b$ \qquad
$\lim_{n \to inf} \frac{a_{n}}{b_{n}} = \frac{a}{b}$
\end{center}

\section{Forme di indeterminazione}

In alcuni casi è impossibile stabilire il comportamento di un limite di una serie, questo avviene nelle forme di indeterminazione.

Ad esempio:

\begin{center}
$\frac{\infty}{\infty}$
\qquad $\frac{0}{0}$
\qquad $1^{\infty}$
\qquad $0^{\infty}$
\qquad $\infty^{\infty}$
\qquad $\infty - \infty$
\qquad $0 * \infty$
\qquad $0^{0}$
\end{center}

\section{Successioni definite per ricorrenza }

Una successione è definita per ricorrenza se è data nella forma:

$$ \begin{cases}
x_1 = a \\
x_{n+1} = F(n, x_n) & \mbox{con }n > 0
\end{cases} $$

\section{Limiti delle successioni elementari}

\section{Successioni asintotiche}

Due successioni ${a_{n}}$ e ${b_{n}}$ sono asintotiche se ${b_{n}} \neq 0$ definitivamente e $$\lim \frac{{a_{n}}}{{b_{n}}} = 1$$ L'asintoticità si indica con $\{a_{n}\} \sim \{b_{n}\}$

\subsection{Proprietà derivate}

${a_{n}} \sim {b_{n}} \leftrightarrow {b_{n}} \sim {b_{n}}$

Se $\{a_{n}\} \sim \{a'_{n}\}$ e $\{b_{n}\} \sim \{b'_{n}\}$ allora $$\lim \frac{\{a_{n}\}}{\{b_{n}\}} = \lim \frac{\{a'_{n}\}}{\{b'_{n}\}}$$


\section{Numero di Nepero}

$e$ (circa 2,71828) è un numero irrazionale: non può essere espresso come frazione o come numero decimale periodico.

In particolare abbiamo visto come $e$ sia il limite della successione 
$$\{(1+\frac{1}{n})^{n}\}$$
 
Per ogni $k \in \mathbb{N}$ vale infatti che $ (1+\frac{1}{k})^{k} < e < \sum\limits_{h=0}^{k} \frac{1}{h} + \frac{1}{2^{k-1}}$

\subsection{Limiti che si deducono da e}

Sia $\{a_{n}\}$ divergente (e quindi con il limite a $+\infty$) allora $$\lim (1+\frac{1}{a_{n}})^{a_{n}} = e$$

\section{Successioni infinitesime}

Una successione $\{x_{n}\}$ si dice infinitesima se $\lim x_n = 0$.

Sia $\epsilon_n$ una successione infinitesima a termini positivi. Allora:

\begin{itemize}
\item $\sin \epsilon_n \sim \epsilon_n$
\item $1 - \cos \epsilon_n \sim \frac{1}{2} (\epsilon_n)^2$
\item $\lim (1+\epsilon_n)^\frac{1}{\epsilon_n} = e$
\item $\log (1 + \epsilon_n) \sim \epsilon_n$
\item $e^{\epsilon_n} - 1 \sim \epsilon_n$
\item $(1+\epsilon_n)^\alpha - 1 \sim \alpha \cdot \epsilon_n$
\end{itemize}
