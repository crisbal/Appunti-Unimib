\newcommand*\dif{\mathop{}\!\mathrm{d}}

\chapter{Equazioni differenziali}

Le equazioni differenziali sono equazioni in cui l'incognita è una funzione e in cui sono presenti una o più derivate della funzione incognita.

Ad esempio $f'(x) + f(x) = x$. Cioè tutte le funzioni che sommate alla propria derivata primma danno come risultato $x$

\textbf{Ordine}: Il massimo ordine di derivazione che compare in un' equazione differenziale.

\begin{example}
$f'(x) = x$

Quali sono le funzioni la cui derivata prima è $x$? 

Tutte le funzioni $\frac{x^2}{2} + c = \int x \dif x$
\end{example}

\section{Tipologie e metodi risolutivi}

\subsection{Equazioni differenziali elementari}

In questi casi basta integrare $n$ volte, dove $n$ è l'ordine dell'equazione differenziale.

\begin{example}
$$
y' = 3e^{2x} &
y = \int 3e^{2x} \dif x &
y = \frac{3}{2} e^{2x} + c
$$
\end{example}

\begin{example}
$$
y'' = 2-\cos(x) &
y' = \int 2-\cos(x) \dif x &
y' = 2x-\sin(x) + c &
y = \int 2x-\sin(x) + c_1 \dif x&
y = x^2-\cos(x) + c_1x + c&
$$
\end{example}

\subsection{Equazioni differenziali a variabili separabili}

$$y' = f(x) \cdot g(y)$$

Per risolverle:

\begin{enumerate}
\item Separare le variabili
\item Integrare ciascun membro
\item Ricavare y(x)
\end{enumerate}

\begin{example}
$$
y' = y^2 \ln(x) &
\frac{\dif y}{\dif x} = y^2 \ln(x)
\frac{\dif y}{y^2} = \ln(x) \dif x
$$
Posso spezzare $\dif x$ e $\dif y$.

Ora integro.

$$
\int \frac{\dif y}{y^2} = \int \ln(x) \dif x &
\frac{-1}{y} = x \ln(x) + c
$$

Esplicito la $y$

$$y(x)  = \frac{1}{x\ln(x)-x+c}$$
\end{example}

Se trovo un $t$ tale che $g(t) = 0$ allora $y(x) = t$ è anche soluzione.

\subsection{Equazioni differenziali lineari}

Vediamo per ora quelle del primo ordine, cioè nella forma:

$$y'(x) + a(x)y(x) = f(x)$$

Come procedere:

\begin{enumerate}
\item Calcolare la primitiva di $a(x) = A(x)$
\item Moltiplico entrambi i membri per $e^A(x)$. Quindi a sinistra ho $[y(x)e^{A(x)}]'$
\item Integro entrambi i membri. $y(x)e^{A(x)} = \int f(x)e^{A(x)} \dif x +c$
\item Moltiplico sia a destra che a sinistra per $e^{-A(x)}$. Ottengo $y(x) = e^{-A(x)} \int f(x)e^{A(x)} \dif x +c$

\begin{example}
$$y'(x)-xy(x)=2x$$
$a(x) = -x$, $f(x) = 2x$

Calcolo $\int a(x) = \frac{-x^2}{2}$ 1

Moltiplico tutto per $e^{\frac{-x^2}{2}}$

Ottengo quindi

$$y(x)\cdot e^{\frac{-x^2}{2}} = \int 2x \cdot e^{\frac{-x^2}{2}} \dif x + c$$

Moltiplico per $e^{-A(x)}$.

Ottengo $$y(x) = -2+c\cdot e^{\frac{x^2}{2}}$$
\end{example}

\textbf{Soluzione generale}

$$
y(x) = e^{-A(x)} \int f(x) e^{A(x)} \dif x + c \cdot e^{-A(x)}
$$

\section{Problema di Cauchy}

Un sistema del tipo 
$$
\begin{cases}
\text{Equazione Differenziale} &
\text{Condizioni Iniziali}
\end{cases}
$$

\begin{example}
$$
\begin{cases}
y' = -e^{-x} &
y(0) = 3
\end{cases} 
$$
$$
y = e^{-x} + c 
$$
Quindi sostituisco $y=f(x)$ e $x=0$
$$
3 = e^{-0} + c &
c = 2
$$

Quindi $y(x)=e^{-x}+2$ è la soluzione al problema di Cauchy.
\end{example}