\chapter{Sistemi Lineari}

\section{Equazione lineare}

\begin{definition}
Si chiama equazione lineare su $\R$ ogni equazione del tipo
$$ a_1x_1 + a_2x_2 + \ldots + a_nx_n = b$$
\end{definition}

$x_1, x_2, \ldots, x_n$ sono dette le variabili dell'equazione lineare.

$a_1, a_2, \ldots, a_n$ sono detti i coefficienti dell'equazione lineare.

$b$ è il termine noto.

\begin{definition}
La $n$-upla ordinata $(k_1, k_2, \ldots, k_n)$ è detta soluzione del sistema lineare se $$ a_1k_1 + a_2k_2 + \ldots + a_nk_n = b$$
\end{definition}

\section{Sistema lineare}

\begin{definition}
Un sistema di $m$ equazioni lineari in $n$ incognite è detto sistema lineare.
\end{definition}

\begin{definition}
Un sistema lineare è detto omogeneo se tutte le $m$ equazioni lineari che lo compongono hanno termine noto $b=0$.
\end{definition}

È possibile rappresentare i sistemi sotto forma di matrici. In questo modo sarà possibile scrivere il sistema come $A \cdot X=B$.

\section{Soluzioni di un sistema lineare}

\begin{definition}
La $n$-upla ordinata $(k_1, k_2, \ldots, k_n)$ è detta soluzione del sistema se essa soddisfa tutte le $m$ equazioni del sistema.
\end{definition}

\begin{definition}
Ogni soluzione è detta soluzione particolare del sistema.
\end{definition}

\begin{definition}
Tutte le soluzioni del sistema sono dette soluzioni generali.
\end{definition}

\section{Teorema di Rouché - Capelli}

\begin{theorem}
Un sistema lineare ha soluzioni se e sole se il rango della matrice incompleta $A$ è uguale al rango della matrice completa $AB$.
\end{theorem}


\section{Regola di Cramer}

Sia dato un sistema lineare di $n$ equazioni ed $n$ incognite.


\begin{theorem}[Teorema di Cramer]
Un sistema lineare ha una e una sola soluzione se il determinante della matrice incompleta $A$ è diverso da 0.
\end{theorem}

\begin{theorem}[Regola di Cramer]
Preso un sistema lineare con $\det(A) \neq 0$ , la componente $i$-esima dell'unica soluzione $(k_1, k_2, \ldots, k_n)$ di tale sistema è data da:

$$k_i = \frac{det(A^1,A^2,\ldots,A^{i-1},B,A^{i+1},\ldots,A^n)}{det(A^1,A^2,\ldots,A^{i-1},A^i,A^{i+1},\ldots,A^n)}$$
\end{theorem}

\section{Sulle soluzioni}

\begin{example}
Prendiamo un sistema lineare con determinante uguale a 0.

Non possiamo applicare la regola di Cramer.

Calcoliamo quindi il rango della matrice completa e della matrice incompleta.

Supponiamo siano entrambi uguali, quindi per Rouché-Capelli il sistema ha soluzioni.

Selezioniamo le equazioni linearmente indipendenti del sistema (il rango ci dice quante sono, basta trovare una sottomatrice $r \times r$ con determinante diverso da zero).

Dal nuovo sistema parametrizziamo (spostiamo le incognite aggiuntive in $B$) e risolviamo con Cramer, ottenendo infinite soluzioni parametriche.
\end{example}


\subsection{Metodo di Gauss}

Si riduce la matrice completa a scala tramite operazioni elementari.

Verifico se il sistema ha soluzioni: cioè applico Rouché-Capelli, guardando se il rango di $A$ è uguale al rango di $AB$

Se esistono, trovo le solzioni: diventerà particolarmente semplice trovarle vista la matrice ridotta a scala.

\subsection{Sistemi lineari omogenei}

\begin{property}
In un sistema lineare omogeneo il rango della matrice incompleta $A$ è uguale al rango della matrice completa $AB$.
\end{property}

\begin{property}
Se il rango della matrice completa è uguale al numero di incognite allora il sistema ha, per il teorema di Cramer, una ed una sola soluzione, quella nulla.
\end{property}

\begin{property}
Se il rango $r$ è minore del numero di incognite $n$, allora il sistema ammette infinite ($\infty^{n-r}$) soluzioni, ottenute parametrizzando $n-r$ incognite.
\end{property}